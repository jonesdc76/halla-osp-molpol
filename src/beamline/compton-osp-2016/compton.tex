%\chapter{Hall A Compton Polarimeter}
\label{sec:compton}
%\footnote{Author: S.~K.~Nanda \email{nanda@jlab.org}}

\vspace{40pt}
\begin{center}
{\bf\LARGE Hall A Compton Polarimeter}
\end{center}
\vspace{40pt}
\newcommand{\dis}{\displaystyle}
\newcommand{\nn}{\nonumber}
\newcommand{\aexp}{A_{exp}}
\newcommand{\araw}{A_{raw}}
\newcommand{\ath}{A_{{th}}}
\newcommand{\ac}{A_{{c}}}
\newcommand{\af}{A_{{F}}}
\newcommand{\pe}{P_{{e}}}
\newcommand{\pg}{P_{\gamma}}
\newcommand{\np}{N^{+}}
\newcommand{\nm}{N^{-}}
\newcommand{\kp}{k^{\prime}}
\newcommand{\krs}{k^{\prime s}_{r}}
\newcommand{\sigrs}{\sigma^{s}_{r}}
\newcommand{\sigc}{\sigma_{c}}
\newcommand{\kmax}{k^{\prime}_{max}}
\newcommand{\wer}{\omega_R}
\newcommand{\wel}{\omega_L}

\section {Introduction}
\label{sec:compton_introduction}
The Hall A Compton
Polarimeter provides  electron beam polarization measurements in a continuous and non-intrusive manner
using  Compton scattering of polarized electrons from polarized photons. A schematic layout of the
Compton polarimeter is shown in Fig.\ref{fig:compton_setup}. The primary features of the Compton polarimeter are:
\begin{enumerate}
\item A vertical magnetic chicane with  four dipole magnets to transport  the CEBAF electron beam to the Compton Interaction Point (CIP).
\item A high-finesse Fabry-Perot (FP) cavity serving as the photon target, located at the lower straight section of the chicane with the cavity axis at  an angle of 24~mr with respect to  the electron beam.  
\item An electromagnetic calorimeter to detect the  back-scattered photons.
\item A Silicon micro-strip electron detector to detect the recoil electrons, dispersed  from the 
primary beam by the third dipole of the chicane.
\end{enumerate}

 \begin{figure}[htp]
    \begin{center}
        \includegraphics*[angle=0,width=\textwidth]{compton_setup}
    \end{center}
    \caption[compton:Schematic  layout]{
            Schematic layout of the  Compton polarimeter
            }
    \label{fig:compton_setup}
 \end{figure}

The electron beam polarization is deduced from the counting rate asymmetries of 
the detected particles. The electron and the photon arms provide redundant measurement of the electron beam polarization.

In the recent years the Compton polarimeter has undergone a major upgrade\cite{compton_upgrade} to green optics, in order to improve the accuracy of polarimetry  for  high precision parity violating experiments at lower beam energies. While maintaining much of the 
the existing infrastructure of the  Saclay design, the green laser upgrade replaced  the original low power 1064 nm  FP cavity  with  a higher power 532~nm system. In addition, the electron detector, photon calorimeter, and data acquisition system  were also  upgraded to achieve beam 
polarimetry at the level of  1\% accuracy, down to $\sim$1~GeV beam energy. The new systems have been  operating  successfully in Hall A beam line 
with  several  kW of cavity power for the past few years. As part of the  12~GeV upgrade of CEBAF, the Hall A Compton polarimeter has been reconfigured  to accommodate the 11~GeV electron beam available to Hall A. The primary changes to the Compton Polarimeter for the 12 GeV Upgrade are:
\begin{itemize}
\item Reduction of the Chicane displacement from 300 to 215 mm.  The change in geometry allows the 11~GeV beam to be transported through the existing dipole magnets while necessitating the raising of the two middle dipole magnets, the optics table, and the photon detector by 85~mm.
\item Increase in  the electron arm acceptance to allow detection of Compton edge in the electron detector with green laser photons.
\item Synchrotron radiation blocker for the electron detector in the straight through beam line after the first chicane dipole. 
\item Suppression of synchrotron radiation background for the  photon calorimeter with addition of field plates to all four dipole magnets that soften the fringe fields seen by the photon detector.
\end{itemize}

The installation of the 12 GeV Compton Polarimeter Upgrade has been completed and the new configuration commissioned during the DVCS experiment running from 2014-2016.


\section {Principle of Operation}
\label{sec:compton_principle}
The Compton effect, light scattering
off electrons, discovered by Arthur Holly Compton (1892-1962), Nobel
prize in Physics, 1927, is one of the cornerstones of the wave-particle
duality. Compton scattering is a basic process of Quantum Electro-Dynamic (QED),
the theory of electromagnetic (EM) interactions.
During 50's and 60's, the QED theoretical developments allow Klein and Nishina to compute accuratly
the so-called Compton interaction cross section. Experimental physicists performed serveral experiments which
are in good agerement with the predictions. This is now a well established
theory, and is thus natural to use the EM interaction, such as Compton
scattering, to measure experimental quantities such as  polarization of an electron beam .
\\

Many of the Hall A experiments of Jefferson Laboratory using a polarized electron beam require a
measurement of this polarization as fast and accurate as possible. Unfortunately the standard
polarimeters, like M{\o}ller or Mott,
require the installation of a target in the beam. Therefore, the polarization
measurement can not to be performed at the same time as the data taking because the beam, after the
interaction with the target, is misdefined in terms of polarization, momentum and position. Another
physical solution has to be found  in order to permit a non-invasive polarization measurement of the
beam. This is the primary motivation for Compton Polarimetry.
\\

This physical process 
%schematically illustrated in Fig.\ref{fig:compton_schematic},
is well described by QED. The cross sections of the polarized
electrons scattered from  polarized photons as a function of their energies and scattering angle
can be precisiely calculated.
The cross sections are  not equal for parallel and anti-parallel orientations  of
the electron helicity and photon polarization.
The theoretical  asymmetry $A_{th}$ defined as
the ratio of the difference
over the sum of these two cross sections is then the analyzing power of the process.
With the kinematical parameters used at JLab, the mean value of this analyzing power
is of the order of few percent.
\\

The polarization of the Jefferson Lab electron beam is flipped up to 2000 times per second. Upon interation
with a laser beam of known circular polarization,
an asymmetry, $A_{exp} = \frac{\np-\nm}{\np+\nm}$,
in the Compton scattering events $N^{\pm}$ detected at opposite helicity.
In the following, the events are defined as count rates normalized
to the electron beam intensity within the polarization window.
The electron beam polarization is extracted from this
asymmetry via % \cite{Prescott:1973ek}

\begin{equation}
\dis
\pe = \frac{A_{exp}}{\pg A_{th}},
\label{eq:compton_def}
\end{equation}

\noindent
where $\pg$ denotes the polarization of the photon beam.
The measured raw asymmetry $A_{raw}$
has to be corrected for
dilution due to the background-over-signal ratio $\frac{B}{S}$,
for the background asymmetry $A_B$ and for any helicity-correlated
luminosity asymmetries $A_F$, so that $A_{exp}$ can be
written to first order as

\begin{equation}
\dis
A_{exp} =  \left( 1 + \frac{B}{S} \right) A_{raw} - \frac{B}{S} A_B
+ A_F.
\label{eq:compton_aexp}
\end{equation}

\noindent The polarization of the photon beam can be reversed with a
rotatable quarter-wave plate, allowing asymmetry measurements
for both photon states, $A_{raw}^{(R,L)}$.
The average asymmetry is calculated as

\begin{equation}
\dis
A_{exp}  =  \frac{\wer A_{raw}^R - \wel A_{raw}^L}{\wer + \wel},
\label{eq:compton_asyRL}
\end{equation}

\noindent where $\omega_{R,L}$ denote the statistical weights of the
raw asymmetry for each photon beam polarization.
Assuming that the beam parameters remain constant over the
polarization reversal and that
$\wer \simeq \wel$, false asymmetries cancel out such that



\begin{eqnarray}
\dis
A_{exp} & \simeq & \frac{A_{raw}^R - A_{raw}^L}{2}  (1 + \frac{B}{S}).
\label{eq:compton_adil}
\end{eqnarray}

Using a specific setup, the number of Compton interactions can be measured for each incident
electron's helicity state (aligned or anti-aligned with the propagation direction). These numbers are
dependent on process cross sections, luminosity at the CIP and time of the experiment.
To first order, assuming the time and luminosity are equal for the both electron helicity states, the
counting rate asymmetry is directly proportional to the theoretical cross section asymmetry.
The proportionnality factor is equal to the values of the photon circular polarization $\pg$
multiplied by the
electron polarization $\pe$, so that measuring the photons polarization and experimental asymmetry, calculating theoretical asymmetry,
one can deduce the electron beam polarization. One electron out of a billion is interacting with
the photon beam which means 100000 interactions per second. So as only few incident electrons
are interacting, these polarization measurements are completly non-invasive for the electron beam
in term of positions, the orientations and the physical
characterictics of the beam at the exit of the polarimeter.
The backward scattering angle of the Compton
photons being very small, the first priority is to separate these particles from the beam using a
magnetic chicane. The energy of the backward photons will be measured by an electromagnetic calorimeter. 
At low energies, a single GSO crystal is used while at higher energies a higher density crystal is required.
At present, a 4 block array of lead-tungstate crystals is under investigation. Both the GSO and lead-tungstate
detectors are assembled and maintained  by Carnegie-Mellon University. The third dipole of the chicane, coupled to the electrons
detector, will be used as a spectrometer in order to measure the scattered electron momentum. To perform
a quick polarization measurement, the photon flux has to be as high as possible. A Fabry-Perot Cavity,
consisting of a pair of multi-layer concave mirrors with very high reflectivity, will amplify this flux to a factor
greater than 10,000. The 15 meter long Compton Polarimeter has been installed in the last linear section
of the arc tunnel, at the entrance of Hall A.

\section{Description of Components}
\label{sec:compton_comp}
As  shown in Fig.\ref{fig:compton_setup},
the Compton polarimeter consists of four major subsystems and associated data acquisition system.  Illustrated in Fig.\ref{fig:compton_geometry} are the  geometrical dimensions for the plan and elevation view of the various elements of the 12 GeV Compton polarimeter. Shown in  Fig.\ref{fig:compton_pic} is a  view of the completed Compton polarimeter from the first chicane dipole end, after the 12 GeV Upgrade. 

The subsystems of the Compton Polarimeter  are described below:

 \begin{figure}[htp]
    \begin{center}
        \includegraphics*[angle=90,width=\textwidth]{compton_geometry}
    \end{center}
    \caption[compton:Geometry  layout]{
            Plan and elevation geometrical views of the 12~GeV Compton polarimeter
            }
    \label{fig:compton_geometry}
 \end{figure}

 \begin{figure}[htp]
    \begin{center}
        \includegraphics*[angle=90,width=\textwidth]{compton_pic_1}
    \end{center}
    \caption[compton:compton  pic]{
            Image of the Compton Polarimeter viewed downstream  from the fisrt chicane dipole. The laser hut containing the optical setup is in the  background.}
    \label{fig:compton_pic}
 \end{figure}

\subsection{Magnetic Chicane}
\label{sec:compton_chicane}

The Compton magnetic chicane, illustrated in Fig.\ref{fig:compton_chicane}, consists of 4 dipoles (1.5 T maximum field, 1 meter magnetic length) here after called D1,2,3,4.

\begin{figure}[htp]
    \begin{center}
        \includegraphics*[angle=-90,scale=0.9]{compton_chicane}
    \end{center}
    \caption[compton:chicane schematic]{
            Schematic layout of the beamline elements along the compton chicane area.
            }
    \label{fig:compton_chicane}
 \end{figure}
(D1,D2) deflect  the electrons  vertically down to steer
the beam through the Compton interaction point (CIP) located at the center of
the optical cavity. After the CIP, the electrons are vertically up deflected (D3,D4) to reach
the Hall A target. The scattered electrons are momentum analyzed by the third dipole and
detected thanks to 4 planes of silicon strips.
The magnetic field is scaled with the beam energy, ensuring the same vertical deflection at the CIP,
up to 11 GeV electrons for 1.5 T field. The parameters of the Chicane are as follows:
\begin{itemize}
	\item The longitudinal magnetic length on the axis of (D1,MMC1P01) and
	(D2,MMC1P02) is 1000 mm.
       \item The distance between the geometrical axis of the
        dipoles (D1,MMC1P01) and (D2,MMC1P02) in the longitudinal plane is 5400 mm
	\item The distance between the beam entry axis in (D1,MMC1P01) anfd the
        beam exit axis in (D2,MMC1P02) in the bending plane (vertical axis), also know as the chicane displacement, is 215 mm.
       \item The bending angle is 2.35$^{\circ}$
\end{itemize}

With higher energy of the 12~GeV Upgrade, synchrotron radiation in the Compton chicane increases dramatically both in flux and energy leading to  dilution of the Compton scattering signal in the detectors. The synchrotron radiation can be suppressed with the  addition of  passive iron plates in the fringe field region of the dipole magnets to reduce the magnetic field seen by the  detectors, thus reducing synchrotron radiation background  to manageable level. Shown in Fig.~\ref{fig:compton_SRS} is a schematic representation of the synchrotron radiation background and its suppression scheme. Dipole magnet D1 poses a potential source of synchrotron radiation for the electron detector via  the straight-through beam line, where as D2 and D3 produce similar background for the photon detector.  These radiations  will be softened with the addition field plates and reduced in flux with absorbers. Dipole magnets D1-D4 have been modified with fringe field plate P1-P4. All four field plate pairs  are identical in geometry, thus preserving the symmetry of the chicane as before the upgrade.  New field integrals were measured after the installation of the fringe plates. The EPICS control database for the magnets have been updated with the new field maps.

A new valve, VBV1P01B, acts as the synchrotron radiation absorber for the straight through beam line. Lead and/or Iron absorbers, matched to the beam energy, are installed external to the scattered photon beam line, for the photon detector.
\begin{figure}[htp]
    \begin{center}
        \includegraphics*[angle=0,width=0.9\textwidth]{compton_SRS}
    \end{center}
\caption{ Illustration of synchrotron radiation suppression scheme with fringe field modifying field plates P1-P4, attached to dipole magnets D1-D4. A combination of reduced magnetic field seen by the  detectors and absorbing material attenuates synchrotron radiation flux to negligible  levels. }
    \label{fig:compton_SRS}
 \end{figure}

\obsolete{
\begin{figure}[htp]
    \begin{center}
        \includegraphics*[angle=0,scale=0.32]{compton_dipole_mcp1p01}
    \end{center}
    \caption[compton:dipole]{
            The first Compton Polarimeter chicane dipole magnet MCP1P01,  identical to the the other three dipole magnets  MCP1P02-4, with
appropriate warning red beacon and sign indicating the presense of magnetic field.}
    \label{fig:compton_dipole}
 \end{figure}
}

\begin{figure}[htp]
    \begin{center}
        \includegraphics*[angle=0,scale=0.32]{compton_D2_field_plates}
    \end{center}
    \caption[compton:Field plates]{
            Field plates P2, as  installed on the second chicane dipole D2.  All four dipoles have identical set of field plates installed.}
    \label{fig:compton_field_plates}
 \end{figure}

\subsection{Optics table}
\label{sec:compton_optics}
A high-finesse Fabry-Perot cavity housed on a optics table serves the role of the photon target.
The optical setup consists of four parts:
\begin{figure}[htp]
    \begin{center}
        \includegraphics*[angle=0,width=\textwidth]{compton_optics_green}
    \end{center}
    \caption[compton:Optics Table]{
            Optics setup  of the  Compton polarimeter
            }
    \label{fig:compton_optics}
 \end{figure}

\begin{enumerate}
\item Green Laser operating at 532~nm wavelength generating up to 3~W power,
\item Input optical transport  form the laser beam to the cavity to optimize laser beam size and polarization,
\item The resonant Fabry-Perot cavity that delivers more than 10~kW of circularly polarized green light
\item Optical devices to measure the circular polarization of the photons at the exit of the cavity
\end{enumerate}


The layout of the optical setup is shown in Fig.\ref{fig:compton_optics}.
Details of the resonant Fabry-Perot cavity for
Compton polarimetry can be found here.\cite{compton_IR_cavity_pub}.
 Expert operations of the initial tuning of lasers and optics, which is beyond the scope of routine operations described in this document,  is governed by a separate Laser Standard Operations Procedure\cite{compton_LSOP}.

\begin{figure}[htp]
    \begin{center}
        \includegraphics*[angle=0,width=\textwidth]{compton_fp_cavity}
    \end{center}
    \caption[compton:FP Cavity]{
            The new 532 nm high finesse Fabry-Perot resonating cavity installed on the optics table inside the laser hut in the Hall A Compton Polarimeter.
            }
    \label{fig:compton_cavity}
 \end{figure}

\subsection{Photon Detector}
To detect the Compton backscattered photons, an electromagnetic calorimeter is used. For low energy measurements, the calorimeter
consists of a single GSO crystal, 60 mm in diameter and 150 mm in length, coupled to a single photomultiplier tube. Higher energy
measurements will require a different crystal; the ideal choice is still under study, but tests have been performed most recently
with a 2x2 array of lead tungstate crystals (also coupled to a single photomultiplier tube). The calorimeter is installed
(Fig.~\ref{fig:compton_gdet})  just behind the third  dipole of the chicane. The backscattered photon are transported to the
calorimeter via a telescoping  beam pipe with a maximum diameter of  1.5~inch. The beam pipe is terminated with a vacuum window and
a lead collimator with configurable absorbers to stop soft photons including synchrotron radiation. This configuration
provides adequate acceptance from  1 to 11 GeV.  

In addition, a pair of finger scintillator and iron converter combination, arranged in an XY configuration, are installed in front
of the calorimeter. The entire assembly is mounted on a remote controlled motorized  table with vertical and horizontal motion
capabilities. The moving  mechanism is used to scan for the peak of the back-scattered photons using the finger scintillators.

\begin{figure}[htp]
    \begin{center}
        \includegraphics*[angle=0,scale=0.75]{compton_gdet}
    \end{center}
    \caption[compton:photon detector]{
            View of the Compton photon GSO Calorimeter.}
    \label{fig:compton_gdet}
 \end{figure}


\subsection{Electron detector}
\label{sec:compton_edet}

The electron detector is made up of 4 planes of Silicon micro-strips composed
of 192 strips each. The micro-strips have 240~$\mu$m pitch (200~$\mu$m Silicon, and 40~$\mu$m spacing), on a  500~$\mu$m 
thick Silicon substrate, manufactured by Canberra systems. The planes are staggered by 80~microns to allow for better
resolution. Shown in Fig.~\ref{fig:compton_edet_strips_cad} is a schematic view of the electron detector. The detector is mounted  in a vacuum chamber on  a vertically movable shaft. A motion control system moves the detector to the appropriate location for the detection of Compton scattered electrons for a given electron beam energy. The detector can be positioned as close as 4~mm to the primary electron beam in order to allow for low energy Compton polarimetery. The external view of the installed electron detector chamber in the Hall A beam line is shown in  Fig.~\ref{fig:compton_edet} 


\begin{figure}[htp]
    \begin{center}
        \includegraphics*[angle=0,scale=1]{compton_edet_strips_cad}
    \end{center}
    \caption[compton:electron detector]{Schematic view of the 4-plane Silicon-mocrostrip Compton electron detector}
    \label{fig:compton_edet_strips_cad}
 \end{figure}


 Illustrated in Fig.\ref{fig:compton_edet} is a view of the actual electron detector.
Distance between the CIP and the first strip is 5750 mm. We recall that
between the CIP and the end of the Dipole 3 is 2150 mm.
For a beam of 3.362 GeV the Compton
edge is at 3.170 GeV. This corresponds to a deviation of 17 mm. Thus at
this energy, only one half of the Compton spectrum is covered and it extends
to the 13th strip of the first plane. The trigger logic looks for a coincidence
between a given number of plane in a "road" of 2 strips. For each trigger
it outputs a signal check by the Polarimeter DAQ.


 \begin{figure}[htp]
    \begin{center}
        \includegraphics*[angle=0,scale=0.7]{compton_edet}
    \end{center}
    \caption[compton:electron detector]{
            View of the Compton electron detector vacuum chamber installed upstream of the 4th chicane dipole. 
            The vertical motion control system and the front end electronics are mounted on top of the chamber 
            }
    \label{fig:compton_edet}
 \end{figure}


\subsection{Data Acquisition System}
\label{sec:compton_daq}

At present, the Hall A Compton polarimeter relies primarily on the ``integrating mode'' data acquisition developed by
Carnegie Mellon University and JLab for thresholdless, readout of the integrated photon detector signal. This system
is based on a Flash ADC system running at 250~MHz. It can also read out the photon detector in ``sampling'' mode, providing
event-by-event information at lower rates.

A fast, counting DAQ for the both the electron and photon detectors is under development. This DAQ will be based on
JLab-designed modules (flash ADC for the photon detector and custom logic board for the electron detector) and allow
event-mode read out at up to 100 kHz. This DAQ is currently in the testing stages and not yet fully deployed.

The electronics for the Compton DAQ systems are located in two electronics racks  1H75B18 and 1H75B19in Hall A as shown in  Fig.\ref{fig:compton_daq_racks}. The data acquisition systems use CODA\cite{compton_CODA} software for online acquisition.

 \begin{figure}[htp]
    \begin{center}
        \includegraphics*[angle=0,scale=.45]{compton_daq_racks}
    \end{center}
    \caption[compton:electron detector]{
            The Compton Polarimeter data acquisition electronics located in racks 1H75B18 and 1H75B19.
            }
    \label{fig:compton_daq_racks}
 \end{figure}


\section {Operating Procedure }
\label{sec:compton_oper}

The main  computer for the compton polarimeter operations, is compton.jlab.org located
in the central isle of the Hall A counting house. A dedicated console display, labelled as Compton,
is in the backroom. This machine, running RedHat Enterprise Linux, runs the compton data acquisition,
analysis, and the EPICS~\cite{EPICSwww} slow control system. To begin compton polarimeter activity log on to:


\noindent machine:  \mycomp{compton.jlab.org}\\
username: \mycomp{compton}\\
password: \mycomp{*******}(contact Dave Gaskell (6092))\\

All necessary environment variables are automatically defined on logon. Follow the steps below
paying careful attention to ensuring that you have checked the result of each step:


\subsection{DAQ Setup}

\begin{itemize}
\item Go to the CODA desktop and open a new terminal window. Type

\mycomp{\$ startcoda}

All relevant CODA processes are started and you would get the {\bf runcontrol panel}. 

\item Click on the {\bf Connect} button. You will get the  window shown in
\item Click on the {\bf Configure} button and you will get the run-type sub-panel 
\item Click on the {\bf Run type} button and choose "{\bf FADC\_prod}" as the  configuration.

\item Confirm via the "{\bf OK}" button
You should see in the window below the following message "transition
configure succeeded"
\item Click on the "{\bf Download}" Button
\item Click ``Prestart''
\item Start the run by clicking on the "{\bf Start Run}" button\\
You should see the run control display
Check that the following happens:\\

{\it transition Go succeeded}\\
the counting rates distribution\\
the number of events in this run is updating\\
the run status {\it active}\\
the run number updated\\

\item To end a run, click on End Run button to stop the acquisition and answer the questions in the end of run panel.
\end{itemize}



\subsection{Cavity Setup}
Choose the EPICS desktop and in a fresh terminal window and start the EDM~\cite{MEDMwww} EPICS panel 
 by executing  the command:\\

\mycomp{\$ NewTools}\\

Navigate to:  {\bf EDM (HLA Main)}  $\rightarrow$  {\bf JMenu (HLA)}

From the JMenu gui, navigate to: {\bf Hall A}  $\rightarrow$  {\bf Compton} $\rightarrow$  {\bf Controls}


This will open the main EPICS menu for the Compton as shown in Fig.\ref{fig:compton_epicsall}.
 \begin{figure}[htp]
    \begin{center}
        \includegraphics*[angle=0,width=\textwidth]{compton_epicsall}
    \end{center}
    \caption[compton:epics main control]{Compton polarimeter main EPICS control panel}
    \label{fig:compton_epicsall}
 \end{figure}

On the control panel, pull the "OPTICS" menu down. Click on "Optics Table" 

	This will bring up the {\bf OpticTable} EPICS control panel as shown in Fig.\ref{fig:compton_table}. Most functions of the lasers and optical setup may be accessed remotely from this  control panel. 


 \begin{figure}[htp]
    \begin{center}
        \includegraphics*[angle=90,scale=.66]{compton_optics_table}
    \end{center}
    \caption[compton:epics main control]{Compton polarimeter Optics Table  EPICS control panel}
    \label{fig:compton_table}
 \end{figure}

Initial setup and calibrations of the optical setup are done by laser trained experts in the laser hut following the LSOP\cite{compton_LSOP}. For an already tuned up system, routine operations  may be accessed from the simpler "mini optics" control panel as shown in  Fig.\ref{fig:compton_optic_mini} , which can be invoked as follows: \\ 

\noindent pull down the "OPTICS" menu from the main control panel. Click on "Mini Optic" 

\begin{figure}[htp]
    \begin{center}
        \includegraphics*[angle=0,scale=0.8]{compton_optic_mini}
     \end{center}
    \caption[compton:epics mini control]{Compton polarimeter mini optics control panel }
    \label{fig:compton_optic_mini}
 \end{figure}

\begin{itemize}
\item {\bf Switch on the laser}\\

To turn the Laser On .... Click on the Laser On button.\\
        Check LASER STATUS and INCIDENT POWER.\\
        A Laser spot may blink on the CCD control TV screen\\ 
	(second from left among the 4 screens)\\
        and you should see a bright spot on the miror control TV screen labelled "laser." 
	(see Fig.\ref{fig:compton_laser_photo}).
 \begin{figure}[htp]
    \begin{center}
        \includegraphics*[angle=0,width=6cm]{compton_laseroff_photo}
        \includegraphics*[angle=0,width=6cm]{compton_laseron_photo}
    \end{center}
    \caption[compton:laser spot]{TV viewer images of laser spot }
    \label{fig:compton_laser_photo}
 \end{figure}

        If Laser doesn't turns ON most problable problem is an interlock
        fault.
        You need an access in Hall A and check the different parts of the
        laser interlock around the optic table:\\
        Two crash buttons on the left wall, inside and outside the laser hut.
        \\
	
\item {\bf Lock the cavity}\\
	
 To lock the cavity click on the Servo On button shown in Fig.\ref{fig:compton_optic_mini}.
\obsolete{
\begin{figure}[htp]
    \begin{center}
        \includegraphics*[angle=0,scale=0.8]{compton_servo_on}
    \end{center}
    \caption[compton:servo control]{Laser servo control}
    \label{fig:compton_servo_on}
 \end{figure}
}
You should see the cavity locking
on the CCD control TV as in  Fig.\ref{fig:compton_cavity_lock}, and you should have more than 4 kW stored in the optical cavity.
\begin{figure}[htp]
    \begin{center}
        \includegraphics*[angle=0,width=\textwidth]{compton_cavity_lock}
    \end{center}
    \caption[compton:cavity lock]{Compton cavity lock acquisition is indicated on the overhead Cavity Lock monitor with a steady bright green laser beam spot emanating from the exit mirror of the FP cavity}
    \label{fig:compton_cavity_lock}
\end{figure}

If it isn't the case, turn on the Slow Ramp and then Click on the Slow On button shown in 
Fig.\ref{fig:compton_optic_mini}.

    If successful you can turn OFF the Slow Ramp button.\\
    Photons are now ready to meet electrons and give some Compton photons children.\\

    If the cavity still doesn't lock after few minutes with SERVO and Slow Ramp ON:\\
Check the Yokogawa generator in the Compton rack (CH01B00).
Frequency should be 928 kHz, Amplitude 80 mVpp and phase -4 deg.
Pull down the OPTICS menu in the main epics window. Click on "Optic table" and then on "Servo".
The laser servo control panel appears.% as shown in Fig.\ref{fig:compton_servo}
Gain should be close to 167. A too high traking level in the feedback can prevent the
cavity from locking. Bring the "tracking Level" cursor down to low values (0.20 - 0.40)
and try to lock again with Servo and Slow Ramp on.

\item {\bf Unlock the Cavity}

On the EPICS control panel, pull the "OPTICS" menu down. Click on "Mini Optic" 

To unlock the cavity, click on the Servo off button in the panel in Fig.\ref{fig:compton_optic_mini}.
\end{itemize}

\subsection{Photon Calorimeter Setup}

If the HV are off, switch them on.\\

The cards of the COMPTON Polarimeter PMT HV are located in crate \#2.
The High Voltage channel for the Compton polarimeter calorimeter is in slot \# 12 (channel 11), nominal
high voltage is -1600 V. High voltage for 4 diagnostic detectors (on laser table) is located in
slot \# 9, channels 0-3; nominal HV is 1700 V.


\begin{itemize}
\item Login to an adaq computer as the "adev" account.  Then go to
the slow control directory "cd ./slowc" and invoke the Java
GUI as "./hvs BEAMLINE".  This pulls up a self-explanatory
GUI Fig.\ref{fig:compton_hv} which shows the state of the HV cards.  One can turn
the HV on and off, enable and disable specific channels,
set HV values, and read back HV values and currents drawn.
\end{itemize}

 \begin{figure}[htp]
    \begin{center}
        \includegraphics*[angle=0,scale=0.45]{compton_hv}
    \end{center}
    \caption[compton:electron hv]{
            The java GUI for  the control of the Compton Polarimeter High Voltage  Channels.}
    \label{fig:compton_hv}
 \end{figure}

\subsection{Electron Detector Setup}
\begin{itemize}
\item {\bf Turn on the electron detector}\\
        The detector system needs to be powered. In hall A there is an
        electrical box called A-UH-B1 left of the stairs going to the tunnel. 
	%(see Fig.\ref{fig:compton_map1_elec})
        In this box, the main power switch for the electron detector is number 21 (it says electron
        detector on it). It must be on turned ON. In the tunnel,
        there is a crate attached to the wall above the electron detector Fig.\ref{fig:compton_edet_crate}, it also needs to be turned ON.
        When it is ON a red LED is lit (at the right end of the crate). Below this crate there is a black electrical
        box controlling the displacement system. On the left side of this box it should say "Idle".

 \begin{figure}[htp]
    \begin{center}
        \includegraphics*[angle=0,scale=0.4]{compton_edet_crate}
    \end{center}
    \caption[compton:electron detector crate]{
            The Compton electron detector instrumentation crate supplying low voltage, high voltage, motion control, and FSD logic to the electron deterctor.}
    \label{fig:compton_edet_crate}
 \end{figure}

\item {\bf Slow control of the electron detector }\\
To perform operations on the electron detector, go to the panel shown in Fig.\ref{fig:compton_epicsElectron}, from the main Polarimeter EPICS screen
and then choose "Electron Detector". On this screen, 
 active buttons appear in blue
and readback values appear on a yellow background. 
To use the electron detector a high voltage
(120 V) must be applied to polarize the silicon
microstrips and a low voltage must be provided to the preamplifier
circuit board and some threshold must be set for each plane for
the detection of the signals. To do this execute the following operations :

Turn the low voltage ON\\
Turn the high voltage ON. The return value should increase gently to 120.\\
Set thresholds to 35. The return value should read 35.\\
Turn calibration OFF.\\

The electron detector can be put in data
taking position remotely. When the detector is inserted {\bf the chicane must be ON},
when it is being moved {\bf the beam must be OFF too}. If it is not the case the detector will eventually be destroyed.

Click on either {\bf GARAGE} or {\bf COMPTON} depending on where you want to put the 
detector.\\

To make sure the detector is where you want watch the detector move on the TV screen  (there is one in the Hall A counting house and one in the back room) as shown in Fig.\ref{fig:compton_edet_viewer}.\\
	
\begin{figure}[htp]
    \begin{center}
        \includegraphics*[angle=0,scale=0.8]{compton_epicsElectron}
    \end{center}
    \caption[compton:Electron detector control]{Compton electron detector control panel}
    \label{fig:compton_epicsElectron}
\end{figure}

 
\begin{figure}[htp]
    \begin{center}
        \includegraphics*[angle=-90,width=6cm]{compton_edet_viewer}
    \end{center}
    \caption[compton:Electron detector viewer]{Compton electron detector TV viewer in the Counting House backroom}
    \label{fig:compton_edet_viewer}
\end{figure}

	  
\item {\bf Switch on the the Compton chicane}\\

        This procedure is only performed by MCC operators.\\
        
 Before contacting MCC, ensure that the 
electron detector is on the {\bf GARAGE}  position.
Check the status of the electron detector on the video screen.\\

        First of all,
        the Hall A Run Coordinator must request that
        MCC tune the beam through the Compton chicane.
        MCC operators have to apply the section 2 of
        the procedure MCC-PR-04-001\cite{compton_beam_tune} . If necessary
       (after a long shutdown for exemple), let's remind to the operator
       to open valves located on the Compton line.
       
\item {\bf Lock once again the cavity}\\
\end{itemize}

\subsection{CIP Scan}
Perform a vertical scanning of the electron beam inside the magnetic chicane 
in order to find the CIP by maximizing the counting rates in the Photon detector.

        In the case the crossing of the electron and Laser beams
        has been lost, or is not optimal, a "CIP scan" has to be performed.
        By stepping the magnetic field of the
        chicane dipoles, the beam is moved vertically. Step size should be
        small with respect to the laser spot size (\~100 micro m). Here are some
        step sizes corresponding to a {\bf 25 or 100$\mu$m  } vertical displacements versus
        typical beam energies, MCC operator are used to Gauss.cm unit:
\begin{table}[ht]
\begin{center}
\begin{tabular}{|l|l|l|} \hline
step 25$\mu$m & step 100$\mu$m & Energy \\ \hline\hline
20 G.cm & 80 G.cm & 2.2 GeV \\ \hline
40 G.cm &  160 G.cm &  4.4 GeV \\ \hline
60 G.cm &  240 G.cm & 6.6  GeV \\ \hline
80 G.cm &  320 G.cm &  8.8 GeV \\ \hline
100 G.cm &  400 G.cm &  11.0 GeV \\ \hline 
\end{tabular}
\end{center}
\caption[Compton:vertical scan]{Chicane CIP  scan step values for various energies.
}
\label{tab:compton_vscan}
\end{table}
	
Although the procedure is non-invasive for Hall A, let the shift leader know
when you start and finish the scan.

The scan is done in contact with MCC (7047) by checking the online evolution
        of the proton counting rate compton:RATE\_G variable with  {\bf StripTool}.
        As a first pass, one can use bigger step size to locate
        the maximum and then go back to small steps to fine tune the position to determine 
        the optimal Y-position

\begin{itemize}
\item {\bf Compton Orbit Lock}\\
When the CIP scan  procedure is over, come back to the right Y-position and
        ask to the machine operator to turn on the "Compton Orbit Lock"  using  the new Y-position of the beam. Then an
        automatic magnetic feed-back will run to keep the electron beam Y-position
        within 50 microns of this optimal position. 

\obsolete{
\begin{figure}[htp]
    \begin{center}
        \includegraphics*[angle=0,height=0.8\textheight]{compton_spy_bell}
    \end{center}
       \caption[compton:vertical scan]{Vertical scan trace in spy\_acq panel}
        \label{fig:compton_spy_bell}
\end{figure}
}

\item {\bf Beam Off}

Request  MCC operators to switch the beam off, in coordination with the Hall A Run Coordinator.

\item {\bf Insert the Electron Detector }

First of all, the electron beam must
be off (see Hall A run coordinator and call MCC operator)
If it is not the case the detector will eventually be destroyed.
To perform operations on the electron detector. Go to the control panel  shown in 
Fig.\ref{fig:compton_epicsElectron}.\\

Click the {\bf COMPTON} button.\\

To make sure the detector is where you want watch the detector move on the TV screen. 
Finally, request  the MCC operators to switch the beam on.
\end{itemize}

\subsection{Taking data}
This is a list of check points to run Compton. It assumes the polarimeter 
has already been started up as described in the previous sections and  
that a run has just ended and you want to take a 
new one.

Bring up the  following three screens to control the data taking:

\begin{itemize}
\item {\bf EPICS screen}: slow control for the optic table + cavity, the 
photon and electron detectors, beam parameters.
\item {\bf Acq screen}: runcontrol.
 \end{itemize}

Follow the following procedure:
\begin{itemize}
\item Go to the {\bf EPICS screen}, check the cavity is loocked with \~4,000 Watts 
or more. Check that the laser is cycling on and off and that the signal to noise is reasonable.
\item Go to the {\bf Acq screen}. Start a new Compton run once every 1 or 2 hours. When starting a run,
make sure that the event number is incrementing (i.e., the DAQ is not stuck).
\end{itemize}

\obsolete{
\begin{itemize}
\item Check Random, Mouly and central crystal are activated.
\item Check {\bf Raw data rates}. Assuming a trigger rate of 1kHz/muA, 
   prescaler factors should keep the read data rates at the few kHz level.
\item Check the trigger condition in {\bf General Daq setup} (Photon only, 
   e only, coinc, ...). 
\end{itemize}
\item Check the state of the acquisition in the {\bf Acquisition system window} 
of spy\_acq. After an "End run succeded" each module should be in "downloaded" 
state. 
{\it If not, follow error messages displayed in the bottom window. Most of 
the problems are fixed by clicking on Reset or Reboot + Download buttons. Display needs 
some delay to refresh after these  actions. Don't click like crazy on every 
enabled button. If everything is stuck try "restart this window" in the 
spy\_acq menu to refresh the display.}
\item Click on {\bf Start Run} in the {\bf Runcontrol window}. Check that 
each module of the acquisition goes from downloaded to paused and then active 
state.
\item Click on the {\bf Online Counting Rate window} in spy\_acq. Check the rate 
in the central crystal (red curve) is close to the optimal value from the last 
vertical scan (it should be around 1kHz/muA).
{\it If counting rates are low and Beam Drift Alarm keeps ringing, the 
crossing of the electron and Laser beams is not optimal. Stop the run and 
perform a 
vertical scan.}
\item Start the photon polarization reversal by clicking on {\bf Procedure ON} 
in the {\bf Left-Right procedure} frame. Periods of cavity ON and OFF will 
alternate, starting with OFF (bkg measurement).
\item Take a {\bf \~1 hour run}.
\item Before ending a run, fill up the {\bf LogBook window} in spy\_acq (name, 
run type, title). Ensure the Logbook and Checklist buttons are not inhibited
if you want the run to be analysed and stored in the electronic logbook.
\item Click on {\bf End Run} in the {\bf runcontrol window}. Look for the 
"End of run succeeded" message in the bottom frame.
{\it If End of run failed, go to Acquisition system in spy\_acq and follow 
error messages.}
\item A {\bf yellow window} should pop up for few second after the end of run 
indicating that the run is saved and the online analysis (checklist) is 
runnig.
\item Go to {\bf Web Browser screen} and reload the {\bf Compton logbook web 
page}. Last run should appear on the first line. 
\item By clicking on {\bf more} you access to detailed informations about 
the running conditions as well as to control histograms generated by the 
{\bf checklist} script. This script may take few minutes to run.
{\bf It is important task to check the control histograms after each end of 
run. Quality off the data depends on it.} See section "Control Histograms".
\item Go to first point, {\bf start a new run}. 
\end{itemize}
}

\obsolete{
Two kinds of alarm can turn ON during data taking:

\begin{itemize}
\item {\bf Y Position:} Go to {\bf On Line Counting Rate window} in spy\_acq and check 
the "Beam drift alarm" frame. If the "Average Y" differs to "Y settings" by more than 
50 microns, Alarm is ringing and stop bell button is red. Click on stop bell and 
wait few seconds to see if the position feedback brings Y back to its nominal 
value. If it doesn't, call MCC (7047) and ask them to check if the position 
feedback on Y in the Compton chicane is still running.
If necessary stop the run, perform a vertical scan and re-lock the vertical 
position at the new optimal value.
\item {\bf DAQ system:} If something goes wrong in the DAQ system during data 
taking you should see an effect on the "photon read" counting rate. Go to 
{\bf Acquisition system} window in spy\_acq, click on stop bell button if alarm is 
ringing. End Run in runcontrol window. Follow error messages displayed in spy\_acq 
to fix the problem.
\end{itemize}
}

Any comment about the running conditions, shift summary, ... are welcome to 
help the offline analysis. You can insert them in the Compton electronic 
logbook by filling up the {\bf LogEntry window} when the run is ended. Click on 
{\bf Submit} to dowload your comments in the logbook.

\subsection{Turning off the compton polarimeter}
\begin{itemize} 
\item  Stop the magnetic chicane\\

        This procedure is only performed by MCC operators.
    
        
        The Hall A Run Coordinator must request that
        MCC turn OFF the Compton chicane and resume normal operations.
    
        MCC operators have to apply the section 3 of 
        the procedure MCC-PR-04-001\cite{compton_beam_tune}. Let's remind to the operator
       to close valves located on the Compton line. It is very important
       to keep the best vacuum in the Compton line and avoid
       dust deposit on the high reflectivity mirrors of the cavity
       
\item Set the electron detector to the garage position.

Before resuming normal operations with beam the electron detector to garage position by clicking 
the  {\bf GARAGE} button on the control panel shown in Fig.\ref{fig:compton_epicsElectron}. Failure to do so, could result in damage to the electron detector. \\


To make sure the detector is where you want it to be, watch the detector move on the TV screen (there is one in the Hall A counting house and one in the back room). At the end of its motion, the arrow on the TV screen should point to the OUT position.

\item  Switch off the PMT High Voltage using the HV  control panel (see Fig.\ref{fig:compton_hv}).

\item Unlock the cavity\\
On the EPICS control panel, pull the "OPTICS" menu down. \\
Click on "Mini Optic".\\
To unlock the cavity, click on the Servo off button.\\
\item Switch off the laser\\
On the EPICS control panel, pull the "OPTICS" menu down.\\
 Click on "Mini Optic".\\
 To turn the Laser Off .... Click on the Laser Off button.
 Check LASER STATUS and INCIDENT POWER.\\
 The Laser spots would switch off on the CCD control TV screen
\end {itemize}

\begin{safetyen}{0}{0}
\section {Safety Assessment}
\label{sec:compton_safety}
\end{safetyen}

\begin{safetyen}{10}{15}
\subsection{Magnets}

Particular care must be taken while working in the vicinity of the dipole magnets of the 
Compton polarimeter magnetic chicane, as they can have large currents
running in them producing strong magnetic fields.  All four dipoles are
powered in series from a common power supply. The power supply for
the dipoles is located in the Beam Switch Yard Building (Building 98) with access restricted to authorized personnel only.
While the power supply  is capable of sourcing  a maximum current of 600~A, the nominal operating current for the dipole magnets is 500~A and is not to be exceeded under normal operating conditions.  All electrical connections to the magnets are covered with thick Plexiglas safety cover shields. These shields can  be removed for service only by authorized personnel, with the concurrence of the Hall A work coordinator, following JLab's "Lock Out/ Tag Out" procedures.

As with any other element that can affect the path of the electron beam, the magnets are controlled by the MCC. Status of the magnets are indicated by a 
red light, located over each of the magnets, which is activated via a magnetic field sensitive switch placed on the coils of one of the dipoles. When the magnets are ON, these lights display a flashing red beacon indicating the presence of magnetic field; only authorized personnel for the Compton polarimeter may  work in the immediate vicinity. At full excitation of the magnets, the leakage field from the magnet could exceed 5~Gauss within a six-inch boundary from the physical ends of each magnet. Access to this region by personnel with  medical monitoring electronic devices and/or metallic implants is prohibited, when the magnets are ON.

\subsection{Vacuum System}
The Compton Polarimeter beam line elements contain thin metal bellows in several places. There is a thin vacuum window at the end of the scattered photon beam line, as well. These could rupture if struck with sharp or heavy objects accidentally, leading to an implosion.  While working near these bellows or windows, protective earmuffs and safety  glasses are required. Only authorized personnel for the Compton polarimeter group may  work near these elements. 

\subsection{Lasers}
The primary laser hazards in the optical table of the Compton Polarimeter are 1064~nm infra-red lasers with up to 30~W CW beam, and a  532~nm green laser up to 3~W CW power.
They are housed in the tunnel in a laser hut with light barriers  on all sides to isolate the laser beams  from the  outside world. A flashing yellow beacon installed in the tunnel indicates laser READY on ON status. Three crash buttons are provided in the tunnel for emergency shutdown of the laser. The acces doors to the laser hut are interlocked to the laser power supplies with  door closure switches. In case a laser hut access door is opened accidentally, the interlock system shuts down the lasers. 

All functions of the lasers are remotely controlled and personnel access to the laser hut
is not necessary during routine operation of the Compton Polarimeter. However, in case of repair  
or maintenance work, access to the laser enclosure may be necessary. The safe operating procedure for this laser is described in 
Jeffeson Lab Laser Standard Operating Procedure \cite{compton_LSOP} (LSOP). A copy of the LSOP is available in the tunnel wall next to the laser hut. Only personnel authorized in the LSOP with appropriate eye protection are permitted to access the laser hut. All other workers not trained for lasers, are required to disable the laser power supply following JLab's "Lock Out/ Tag Out" procedures in coordination with the Hall A Work Coordinator.
 
\subsection{High Voltage}

There are up to 25 photomultiplier tubes within the Compton photon detector module. There are also several beam diagnostic scintillation counters dispersed along the Compton Polarimeter chicane.
Each tube is connected to a high voltage power supply located in the beamline instrumentation area. The maximum voltage is 3000 Volts. 
In addition, the electron detector is supplied with up to 350~Volts. Only SHV connectors may be used to connect the high voltage to the detectors. 
The high voltage supply source must be turned off prior to conecting or disconnecting the HV to the  detector element being accessed
for servicing purposes. Only members of the Compton group
are authorized to access the  detectors.\\

\end{safetyen}

\subsection{Authorized Personnel}
The list
of the presently authorized personnel is given in Table~\ref{tab:compton:personnel}.
Other individuals must notify and receive permission from
the primary contact person (see Table~\ref{tab:compton:personnel}) before being authorized to work on the Compton Polarimeter.

\begin{table}[ht]
\begin{center}
\begin{tabular}{|ll|l|l|l|l|c|} \hline
  \multicolumn{2}{|c|}{Name} & Dept. & \multicolumn{2}{c|}{Telephone} &
  \multicolumn{1}{c|}{e-mail} & Comment \\
  \cline{4-5}
    first & last & & JLab &  Cell&  & \\
\hline
 {Dave } & {Gaskell} & JLab   & 6092  &   & gaskelld@jlab.org     & \it Primary     \\
   Alexandre & Camsonne           & JLab   & 5064 &  & camsonne@jlab.org     & \it Alternate \\
 \hline
\end{tabular}  
\end{center}
\caption[compton Polarimeter: authorized personnel]{
   Compton Polarimeter: authorized personnel. }
\label{tab:compton:personnel}
\end{table}

%\bibliographystyle{plain}
\begin{thebibliography}{99}
%\begin{thebibliography}
\bibitem{compton_upgrade} S. Nanda and D. Lhuiellier, Conceptual Design Report  for 
Hall A Compton Polarimeter Upgrade, \url{https://userweb.jlab.org/~nanda/compton/HallA_Compton_Upgrade.pdf}.

\bibitem{compton_IR_cavity_pub} \url{http://hallaweb.jlab.org/compton/Documentation/Papers/nima4592001.pdf}.

\bibitem{compton_LSOP} Jeffeson Lab Laser Standard Operating Procedure for Hall A Compton Polarimeter Laser System,\url{https://jlabdoc.jlab.org/docushare/dsweb/Get/Document-86141/PHY-14-002-LOSP.pdf}.

\bibitem{EPICSwww} Experimental Physics Instrumentation and Control System, \url{http://www.aps.anl.gov/asd/controls/epics/EpicsDocumentation/WWWPages/EpicsDoc.html}.

\bibitem{compton_CODA} CEBAF Online Data Acquisition, \url{http://www.coda.org/}.

\bibitem{compton_beam_tune} Beam tuning with the Hall ACompton Chicane,
\url{http://opsntsrv.acc.jlab.org/ops\_docs/online\_document_files/MCC\_online\_files/HallA\_beam\_delivery\_proc.pdf}
\end{thebibliography}

%%\end{document}

%\JosephZhang{Optics}
%\MartialAuthier{Engineering}
%\NathalieColombel{Mechanical}
%\PascaleDeck{Electronics}
%\AlainDelbart{Optics}
%\DavidLhuillier{Analysis}
%\YvesLussignol{EPICS}
%\DamienNeyret{DAQ}
%\GerardTarte{Electronics}
%\ChristianVeyssiere{Electronics}
