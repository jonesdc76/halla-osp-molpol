%\section[Aerogel \Cherenkov{} Counter]{Aerogel \Cherenkov{} Counters
\chapter[Aerogel \Cherenkov{} Counter]{Aerogel \Cherenkov{} Counters
\footnote{Author: B. B. Wojtsekhowski \email{bogdanw@jlab.org}}
}
\label{chap:hrs-aerogel}

\section{Overview}

There are three aerogel \Cherenkov{} counters available with various indices
of refraction, which can be installed in either spectrometer 
and allow a clean separation of pions, kaons and protons over 
the full momentum range of the HRS spectrometers.
The first counter (AM) contains hygroscopic aerogel%
\footnote{Airglass AB, BOX 150, 245 22 Staffanstorp, Sweden.}
with a refraction index of 1.03 and a thickness of 9~cm. 
The aerogel is continuously flushed with dry CO$_{2}$ gas.  
It is viewed by 26 PMTs (Burle 8854~\cite{BurleInd}).
%\footnote{Burle Industries Inc., 1000 New Holland Ave., Lancaster, PA 17601, USA,
%\url{http://www.burle.com/}}). 
\infolevltone{
A cross sectional schematic of the detector is shown in Fig.~\ref{fig:aero_fig2}.
} %infolev 

For high-energy electrons the average 
number of photo-electrons is about 7.3~\cite{Brash:2002vn}.

The next two counters (A1 and A2) are diffusion-type aerogel counters.
A1 has 24 PMTs (Burle 8854). The 9 cm thick aerogel radiator used in A1
\footnote{ Matsushita Electric Works, {\tt www.mew.co.jp}.}.
has a refraction index of 1.015, giving a threshold of 2.84 (0.803) GeV/$c$ for
kaons (pions). The average number of photo-electrons for GeV electrons
in A1 is $\simeq$~8% 
\infolevtwo{(see Fig.~\ref{fig:diff_performance})}.
\infolevtwo{
\begin{figure}[htb]
\begin{center}
  \includegraphics[angle=-90,width=0.8\textwidth]{diff_aero}
\end{center}
\caption[Number of photo-electrons in A1 and A2 vs particle momenta and 
the amplitude spectra]
{Number of photo-electrons in A1 and A2 vs particle momenta and 
the amplitude spectra.}
\label{fig:diff_performance}
\end{figure}
} %infolev
 The A2 counter has 26 PMTs
(XP4572B1 from Photonis~\cite{PhotonisInd}).
%\footnote{Photonis, Avenue Roger Roncier, Z.I. Beauregard,
%B.P. 520, 19106 Brive Cedex, France, \url{http://www.photonis.com}}).
The aerogel in A2 also hygrophobic has a refraction index of 1.055, 
giving a threshold of 2.8 (0.415) GeV/$c$ for protons (pions). 
The thickness of the aerogel radiator in A2 is 5 cm, producing an average
number of about 30 photo-electrons for GeV electrons.

%\author{G.J. Lolos}
%\address{Dept. of Physics, University of Regina, Regina, SK, S4S 0A2 Canada}

\infolevone{
\section[Mirror Aerogel \Cherenkov{} Counter]{Mirror Aerogel \Cherenkov{} Counter
\footnote{
  $CVS~revision~ $Id: aerogel.tex,v 1.14 2005/04/04 22:27:25 gen Exp $ $
}
\footnote{Author: G.J. Lolos \email{gjlolos@jlab.org}}
}
AM is a silica aerogel \Cherenkov{} counter with a compact reflection mirror design, 
which was dictated by the available space (36.3 cm along the incident particle direction). 
In addition, the high singles rates expected in Hall A are better handled 
with segmented detectors covering the focal plane; 
this requires short pulse decay times.
  Even though the diffusion length in silica aerogel can be quite short for 
low $\lambda$ light generated in the $SiO_2$ radiator \cite{Lippert:1993kt}, 
enough directionality remains in the visible $\lambda$ region, 
where the selected PMTs have good quantum efficiency, to make
light collection with mirrors an attractive and practical alternative. 

An effective segmentation of the aerogel \Cherenkov{} counter, matching the 
segmentation of the trigger scintillators, can be used to separate multiple
tracks through the focal plane and will allow an additional element of
selectivity and track sensitivity in the focal plane instrumentation.  This
means that specific sections of the focal plane can be physically disabled from
the trigger if the experimental conditions require it.  It will also provide
the capability to identify and separate pions and protons traversing the
focal plane trigger scintillators and the vertical drift chambers (VDCs) within
the resolving time of the system (double hits).  
For example, in the off-line analysis, the aerogel counter PMT with the highest 
number of photoelectrons can be matched with the trigger counter and 
VDC information to identify the actual path of a pion, 
thus separating it from a simultaneously detected proton which
has no \Cherenkov{} signature.  
Such a capability of double hit resolution is not possible with diffusion 
\Cherenkov{} counter designs, because the photon collection efficiency does not 
have a strong spatial correlation with the incident particle track within 
the aerogel material. 
 
The segmentation, in addition to supplementing the information
on the individual particle position along the focal plane, couples well
with the desire to increase the active solid angle viewed by the PMTs
in the counter.  Although the photon detection probability is not as directly
proportional to the solid angle covered by PMTs as in the case of a diffusion
box, clearly the larger effective coverage leads to a higher probability
that a photon will end up on a PMT.  Given the divergence of the beam
envelope incident on the aerogel and the diffusion of the light in the low
$\lambda$ region by the aerogel material, an increase in the area covered by
PMTs results in an increase in the number of photons detected.  
As a result, a total of 26 PMTs are used in the counter, as shown in 
Fig.~\ref{fig:aero_fig1}, with minimal spacing between their $\mu$-metal shields 
(2.8 $mm$).  
The total area covered by the PMT photo-cathode windows comprises 72\% of the area 
of the counter opposite the planar parabolic mirrors.  
A cross sectional schematic of the detector is shown in Fig.~\ref{fig:aero_fig2}, 
clearly illustrating the planar parabolic design of the mirror surfaces and their 
relative orientation with respect to the PMTs, and the orientation of the counter 
relative to the central axis of the spectrometers. 
   
The close spacing of the $\mu$-metal shields
%, which is also shown in the
%photograph of figure \ref{fig:aero_fig3}, 
creates dielectric breakdown problems.  The $\mu$-metal
shields are at cathode potential (-2950 $V$) to avoid the capacitive discharge
from a grounded $\mu$-metal shield to the glass of the photo-cathode; discharges
would contribute to the noise level in the PMT and adversely affect their
performance at high operating voltages. This necessitates extra precautions in
order to avoid dielectric breakdown between adjacent shields and between the
shields and the aluminum structure of the counter, which is at ground
potential. The solution was to wrap the outer surfaces of the $\mu$-metal
shields with a high dielectric value (12,000 $V/mm$), thin (0.254 $mm$) Teflon
film\footnote{DuPont Canada Inc., Box 2200, Streetsville, Mississauga, ON L5M 2H3, Canada.}.
  In addition, the PMT housings consist of fiberglass-epoxy composites, 
with added inner and outer skins of 0.0254 $mm$ thick Tedlar\footnotemark[1],
with a further combined insulating value of 3,000 $V$.  
Such a combination of insulating materials eliminates any breakdown or small leakage
current induced noise and simultaneously satisfies all safety requirements.

The final construction of the counter, described in this report, is built
around the two sides of the main (PMT) section. Each section consists of two pieces
of aircraft quality aluminum alloy, with stiffening aluminum rods formed
integrally on the top and bottom.  
The openings for the PMT housings were machined on these structures using CNC 
milling machines to keep tolerances to fine levels.  
The double-walled structure, on both sides of the enclosure, further increases 
the rigidity of the exoskeleton by forming a second ``outer'' wall on each side 
- very similar in configuration to the inner one - and attached to the latter with 
cross-bolt braces.
%, as shown in the photograph of figure\ref{fig:aero_fig3}.
Each end plate is made out of the same aluminum alloy as the side walls, and
also incorporates stiffening lips folded integrally to each plate, one at the
top and one at the bottom.  Each end plate has been provided with inlet and
outlet gas line connections, which will be used to fill the counter enclosure
with dry $CO_2$ gas to protect the silica aerogel from water vapor absorption.
%figures \ref{fig:aero_fig3} and \ref{fig:aero_fig4} show the bottom (tray) sections, 
%and main plus upper (mirror) sections, respectively.  
%The main (middle or PMT) section, in figure \ref{fig:aero_fig4},
%contains the PMTs and provides the strength and rigidity for the whole counter.
%The one piece aluminum end plates are also shown in both photographs. 

All internal surfaces of the detector, except the planar parabolic mirrors
are lined with aluminized 
mylar\footnote{National Metalizing , P.O. Box 5202, Princeton, NJ 08540, USA.} 
to increase the overall reflectivity of the counter.  
The mirrors are made of $45\times 20.5 cm^2$ molded surfaces, formed in one 
rigid structure. 
The rigidity is provided by two layers of carbon fiber epoxy composite backing, 
with a combined thickness of 0.28 $mm$, and a single sheet of mylar with 
thickness 0.127 $mm$. 
The special mylar material was obtained from exposed negative film used in the
cartographic industry, and is of high smoothness and uniformity. One side was
aluminized at CERN, while the other side remains in its exposed negative
(black) state, further adding to the successive light penetration barriers into
the enclosure.  
 
The upper section of the counter containing the mirrors is mounted on its
own aluminum sub-frame, which is bolted to the main frame that houses the PMTs. 
%The upper section, on its own, is shown in the photograph of 
%figure~\ref{fig:aero_fig6}, while
%its configuration when mounted on the main section is shown in
%figure~\ref{fig:aero_fig3}.  
The light and gas sealing action is provided by continuous twin parallel rubber
strips along the joint area, and by Tedlar film of 0.025 $mm$ thickness
covering the top of the outer planar parabolic area. 
%
\begin{figure}[htb]
\begin{center}
  \includegraphics[angle=0,width=0.6\textwidth]{aero_fig1}
\end{center}
\caption[Aerogel:layout]{
 Schematic diagram of the aerogel \Cherenkov{} counter as viewed by the
 incoming particles.  The numbers indicate the sections, 1 to 13, in the
 counter. Each section is viewed by two PMTs, one on the top (T) and one in the
 bottom (B). The labeling carries no significance other than identifying the
 PMTs during the testing phase, as described in the text.
 }
\label{fig:aero_fig1}
\end{figure}

} %infolev
\begin{figure}[htb]
\begin{center}
  \includegraphics[angle=0,width=0.6\textwidth]{aero_fig2}
\end{center}
\caption[Aerogel:mirrors]{
 Cross sectional drawing of the counter, along the particle direction, 
 showing the planar parabolic nature of the mirrors and the geometry of the 
 PMTs.  The joint of the two mirror surfaces in
 the middle of the counter defines the mirror ``ridge''.
 }
\label{fig:aero_fig2}
\end{figure}

\infolevone{

The third major component of the counter is a removable tray where the
silica aerogel is placed. 
The tray occupies the bottom part of the counter and has inner dimensions of 
$195\times 41 cm^2$. 
It is formed by a frame with twin aluminum panels which secure 
the removable frame strung with fishing line in a criss-cross pattern to hold 
the aerogel panels in place.  This ``fish-net'' frame is secured by
screws and is easily removed without disturbing the aerogel panels or requiring
re-stringing.  The bottom of the tray is formed out of a single layer of carbon
fiber epoxy skin (0.127 $mm$ thick) and a layer of aluminized mylar of equal
thickness.  Externally, it is covered by a single layer of Tedlar film to
assure integrity from light penetration; further environmental isolation is
provided by two parallel strips of rubber gasket seals enclosing the
circumference of the tray and containing the feed-through spacers for the
retaining bolts.   The tray is equipped with SMA-type fiber optic feed through
connectors for the gain and timing monitor system, which utilizes fiber
optic cables.  Each fiber illuminates two adjacent PMTs, except for the last PMT
on either side (13T and 13B in Fig.~\ref{fig:aero_fig1}) 
which have their own dedicated fiber. 
The light is generated in a gas plasma discharge 
unit\footnote{Optitron Inc. 23206 S. Normandie Ave. \#8, Torrance, CA 90502, USA.} 
and duplicates the spectrum expected from \Cherenkov{} radiation.  In addition, the
fibers terminate beneath the silica aerogel; thus the light reaching the PMTs
will have the absorption characteristics of real \Cherenkov{} light produced 
in the aerogel radiator. 

Due to the nature of \Cherenkov{} detectors, where only a few photoelectrons (PEs) are
emitted by the photo-cathodes in the PMTs, any extraneous light entering the
enclosure is very troublesome.  As a result of the small number of PEs
expected, the PMTs operate either near to or at maximum high voltage, and
thus at maximum gain.  
As such they can suffer damage if a sudden light leak develops.  
In testing we verified the extreme sensitivity to minute light leaks
across the whole length of the structure because of the mirrored
surfaces inside the enclosure.  
With 26 PMTs operating at maximum gain - and viewing, effectively, a giant mirror - 
sealing the enclosure against single photon penetration requires extra care during 
initial testing and operations.

The PMTs chosen for the counter were Burle model number 8854 with a 127 $mm$ photo-cathode 
diameter
\footnote{Burle Industries Inc., 1000 New Holland Ave., Lancaster, PA 17601, USA.}.
  The PMT amplification electronics have been described in 
Refs.~\cite{Alexa:1995ne,Lolos:1997vz}.  The dynode chain incorporated a
600 $k\Omega$ resistance between the cathode and first dynode instead of the
nominal 300 $k\Omega$.  This generates a $V_{dyn}=885 V$ voltage drop across the cathode to
dynode gap, thus increasing the photo-electron collection efficiency and peak
to valley (P/V) ratio.  This modification has proven successful in
increasing the PE collection efficiency and the single PE resolution.  The
dynode amplification chain also incorporates a 11 $M\Omega$ resistor in series
with the $\mu$-metal shield to eliminate the possibility of electric shock
through careless handling; this high impedance also limits the current drawn
in the unlikely event of a complete dielectric breakdown between the shields
and the aluminum parts of the detector. A schematic diagram of the electronic 
amplification chain is shown in Fig.~\ref{fig:aero_fig7}. 

The operation of the aerogel detector is discussed in Ref.\cite{Brash:2002vn}.

%\begin{figure}[p]
%\includegraphics[angle=0,width=15cm]{blank}
%\caption[Aerogel: structure]{
% Photograph showing the final arrangement of the double sidewall
% structure, and the close spacing of the housings for the PMTs.  The bottom
% section of the counter, with the tray and the aluminized mylar reflector
% lining, is also shown.  In this picture, the particles would be incident from
% the bottom toward the top of the counter.  The upper (mirror) section has been
% removed for clarity.
% }
%\label{fig:aero_fig3}
%\end{figure}

%\begin{figure}[p]
%\includegraphics[angle=0,width=15cm]{blank}
%\caption[Aerogel: middle PMT]{
% Photograph showing the middle (PMT) section with the double sidewall
% structures and the housings for the PMTs, with the top (mirror) section
% attached.  The tray has been removed, and the white tabs on the mirrors are
% pieces of tape holding a temporary protective film in place to prevent damage
% to the mirror surfaces during transportation.  In this figure, the particles
% would be incident from the top toward the bottom of the picture.
% }
%\label{fig:aero_fig4}
%\end{figure}

%\begin{figure}[htp]
%\includegraphics[angle=0,width=15cm]{aero_fig5}
%\caption[Aerogel: reflectivity]{
% Typical reflectivity curve of a mirror as a function of the 
% wavelength, $\lambda$, of the incident light.
% }
%\label{fig:aero_fig5}
%\end{figure}
%
%\begin{figure}[p]
%\includegraphics[angle=0,width=15cm]{blank}
%\caption[Aerogel: upper mirror]{
% A photograph showing the upper (mirror) section with the planar
% parabolic mirrors.  The mirror surface is protected by a vinyl film in this
% picture.  The mirror ``ridge'' separating the counter into two halves is
% clearly seen.  This section fits over the open top of the counter in Fig.\ref{fig:aero_fig3}.
% }
%\label{fig:aero_fig6}
%\end{figure}
%
\begin{figure}[h]
\begin{center}
  \includegraphics[angle=0,width=0.8\textwidth]{scint_fig4}
\end{center}
\caption[Aerogel: amplification chain]{
 Schematic diagram of the electronic amplification chain. The total 
 resistance of 600 $k\Omega$ between the cathode and the first dynode is shown 
 as three 200 $k\Omega$ resistors for sake of clarity. In the actual PC boards, 
 the arrangement is of six resistors of 100 $k\Omega$ each, in order to 
 keep the voltage across each resistor low and avoid surface discharge 
 between the closely packed resistors.
 }
\label{fig:aero_fig7}
\end{figure}

} %infolev

\infolevtwo{
\section{Operating Procedure}

\paragraph{Operating Voltage}

The operating voltage on the PMTs is -2,950 V. This is a near the maximum rated 
voltage and it has been shown to provide high efficiency, good P/V ratio,   
and long PMT life. The overall gain of the PMT is not at maximum, as    
measured by BURLE, since the dynode chain of the 13 dynodes (2nd dynode to   
14th dynode) is kept at a -2,600 V equivalent with the original 300 $k\Omega$   
resistor value between the cathode and 1st dynode. However, the gain is more   
than sufficient to separate single PEs from the pedestal on all ADCs we have   
used so far. It should not be necessary to increase the voltage above the   
recommended value. 

\section{Handling Considerations}

It is generally not advised to open up the counter if the persons involved are
not thoroughly familiar with the assembly and specific component function.
Routine operation does not require any hands on modifications to the detector,
as long as the following operating principles are followed: 

\paragraph{Installation and Removal of PMTs}

The replacement of a PMT or repairs of the electronic amplification chain can   
be accomplished by the removal of that specific PMT-Base combination. Turn the   
HV off on all PMTs and remove the rubber hood covering the base and housing 
interface region. Now remove the three small screws attaching the base to   
the integral housing. Note that the base can only be secured to the housing   
in one specific orientation. 

Carefully slide out the base with the PMT and $\mu$-metal shield mounted as one
unit. Remove the elastomeric ring positioned between the PMT and the   
$\mu$-metal shield. Loosen the nut securing the $\mu$-metal shield to the   
base and carefully apply upward force on the shield while someone else is   
holding onto the base. This will remove the PMT and the $\mu$-metal shield   
from the socket and base, respectively. 

The replacement of the PMT requires experience because it has to be done with the
$\mu$-metal shield installed in, but not secured to, the base. The PMT pins
need to be aligned with the socket pins in a specific geometry, thus the
insertion has to be done by feel and experience. Once the PMT is inserted in
the socket, the $\mu$-metal shield is secured the base with the nut. Make sure
the shield protrudes past the photo-cathode as much as the tapered design
allows. Carefully insert the elastomeric ring between the PMT rim and the
$\mu$-metal shield. This ring supports the PMT and prevents it from sliding 
out of the pins during movement; it also helps seal the interior of the counter
from the outside environment and reduces the $CO_2$ leakage rate. 
Reverse the process for installation.

\paragraph{Installation and Removal of the $SiO_2$ Tray}

{\bf PLEASE NOTE}: The $SiO_2$ aerogel panels are extremely fragile and sensitive to 
water and chemical vapor. Do not handle with bare hands: use clean cotton, or 
another fabric type, gloves instead. Surgical gloves often are contaminated with 
lubricants and are not suitable for this purpose. 

The tray is secured to the main section by hex bolts. The removal of the bolts 
results in the straightforward removal of the tray. There is minimum clearance 
between the tray walls and the main section; as a result, the tray has to 
be removed and installed in a uniform translation with respect to the main 
body. The frame supporting the fish net (or tennis racket) can be removed from 
the tray proper by removal of the two small screws in the middle of the tray 
walls; a tool (hook) is provided for this operation. The $SiO_2$ aerogel 
panels can now be removed or replaced. Reverse the procedure for installation.
The securing bolts do not need to be tightened very much and, although spacers 
are inserted between the rubber strips to prevent damage, care and common sense 
 should be exercised. Light and gas sealing is provided by the 
rubber strips - NOT by brute force.

{\bf WARNING}: After each removal of any components of the counter, check for light 
leaks before turning the HV on at operating values. Even a small light 
leak can destroy the PMTs if they are at -2,950 V! Check for light leaks with 
lights out, using a small portable light and a reduced voltage around -2,000 V.

\section[Diffusion aerogel counters]{Diffusion aerogel counters 
\footnote{Author: B. B. Wojtsekhowski \email{bogdanw@jlab.org}}
}

For a reliable PID of kaons with momenta up to 2.84 GeV/$c$ 
an aerogel detector with a low refraction index of 1.015 (A1) was constructed.
With a low index the light yield is expected to be less, a new design
of the counter was evaluated and optimized. The result is an
average number of 8 photo-electrons. 

For the reliable positive identification of kaons and rejection of protons, 
a large number of photo-electrons is very important. The second  diffusion
aerogel counter (A2) was constructed with an aerogel index of refraction of 1.055. 
With only a 5 cm thickness of aerogel, almost 30 photo-electrons were collected. 
The large collection efficiency was achieved through several design considerations and
use of different type of PMT - XP4572B.

Each detector consists of a tray for the aerogel radiator and a diffusion box
which holds the PMTs. The surface of each box is covered with millipore
paper. Hydrophobic aerogel was used for both detectors, however the boxes
are gas tight so hygroscopic aerogel also can be used. The positive HV 
used in the detector allows the increase of the solid angle viewed by each PMT
and as result the increase of the light collection efficiency. The PMTs don't have
$\mu$-metal shields because the magnetic field at the location of these
detectors  doesn't effect the light collection efficiency.
The schematics of A1 and A2 are shown in figs~\ref{fig:A1_scheme} 
and~\ref{fig:A2_scheme}.

\begin{figure}[htb]
\begin{center}
  \includegraphics[angle=0,width=0.8\textwidth]{A1_scheme}
\end{center}
\caption[The scheme of A1 detector] {The scheme of A1 detector.}
\label{fig:A1_scheme}
\end{figure}
%
\begin{figure}[htb]
\begin{center}
  \includegraphics[angle=0,width=0.8\textwidth]{A2_scheme}
\end{center}
\caption[The scheme of A2 detector] {The scheme of A2 detector.}
\label{fig:A2_scheme}
\end{figure}
%
%
} %infolev

\infolevfour{
The structure of the diffusion box is shown in Fig.~\ref{fig:A2_PMT}.
The picture was taken before installing the millipore paper.

\begin{figure}[p]
\begin{center}
  \includegraphics[angle=0,width=12cm]{a2_PMT_box1}
\end{center}
\caption[The diffusion box of A2 detector]
{The diffusion box of A2 detector.}
\label{fig:A2_PMT}
\end{figure}

Fig.~\ref{fig:A1_inside} shows the view of the A1 counter from the inside.
The semi-spherical photo-cathodes of the 8854 are on left and right sides. 
The white wires on the bottom were installed to prevent the motion of the aerogel blocks 
during detector transportation.

\begin{figure}[htb]
\begin{center}
  \includegraphics[angle=0,width=12cm]{a1_inside}
\end{center}
\caption[Aerogel A1 from inside of the detector ]
{Aerogel A1 detector from inside of the diffusion box. }
\label{fig:A1_inside}
\end{figure}

Fig.~\ref{fig:diff_performance} shows performance characteristics 
of the A1 and A2 counters. 

} %infolev

\begin{safetyen}{10}{10}
\section{Safety Assessment}

The PMTs are under high voltage and care is required when handling any 
components of the counter. As stated earlier on in this report, the insulating 
material between the $\mu$-metal shield and the aluminum exoskeleton far 
exceeds the requirements dictated by the operating voltage. 
In addition, the 11 $M\Omega$ resistor between 
the $\mu$-metal shield and the HV source restricts the current flow below the  
critical 1 $mA$ level. The combination of Tedlar film, Plexiglas composites, 
and injection molded bases are all safe to handle but care should be 
exercised when handling the aluminum parts of the counter or touching the metal 
back plate of the base. It is strongly recommended to ground the aluminum 
exoskeleton of the counter, at several spots to a common ground with the HV 
and signal cable ground. This will further enhance safety and eliminate 
potential ground loops in the unlikely event of a slow, and otherwise difficult 
to diagnose, dielectric breakdown between the $\mu$-metal shield and aluminum 
structure or aluminized mylar of the interior.  

\obsolete{
The PMTs are under high voltage and care is required when handling any 
components of the counter. The body of the counter must be grounded.
Positive polarity of the HV made operation much more simple, but in some
situation the HV can be on the body of counter, for example in case of 
the PMT vacuum failure.
The important requirement is to switch HV off before disconnection or 
connection HV cable to the HV divider. If some of the PMT need replacement the 
HV must be off for all of them and HV supply must be disconnected 
from all PMT.
}
\end{safetyen}

\begin{safetyen}{10}{15}
\section{Authorized Personnel}
\end{safetyen}

The individuals responsible for the operation 
of the aerogel \Cherenkov{} counters are given in Table \ref{tab:agel:personnel}.

\begin{namestab}{tab:agel:personnel}{Aerogel counters: authorized personnel}{%
      Aerogel counters: authorized personnel.}
  \BogdanWojtsekhowski{\em Contact}
\end{namestab}

