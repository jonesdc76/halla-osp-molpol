
%
% Include here Hall dependent operation procedures
%
\newpage
\section{Special Procedures for \HALL}
 All magnets must be ramped-down to zero current prior to entering Hall A during a Controlled Access.
MCC must ramp-down to zero current the SBS and corrector magnets power supplies before the access
takes place. 
%\begin{itemize}

%\item{
%A special tritium target training, SAF110T, is now required for all shift workers.
%}
%
%\item{
%When the tritium target is installed in the Hall A scattering chamber, the Hall shall
%will require badge access.   Each person should badge in seperately.
%}
%
%\item{
%The tritium target must be in the "HOME" position to initiate a a controlled access.
%During a controlled access target may only be moved by system experts.
%}

%\item{
%If all other trainings are up-to-date, the Hall A walk-through, SAF110, can be arranged 
%with Jaiver Gomez or Douglas Higinbotham; though during operations those training will only 
%happen opportunistically.   
%}

% This was part of the tritium experiment.
%\item{
%No one shall enter the fence area around the target or the dump area between the HRS without
%permission from the radiological control group.  
%}

%\end{itemize}

%Further details about the tritium target, as well as, details of the tritium target proceedures
%can be found at: \url{https://wiki.jlab.org/jlab_tritium_target_wiki}.
 
%\EXPTS uses only standard equipment and thus only standard rocedures are required.

%Beyond the information covered in the Hall A standard equipment manual,
%a special operational safety proceedure, OSP-XXXX has been written for this
%experimental period.  OSP-XXXX covers all aspects the tritium target safety
%including the stricted access to the Hall and special training requirements.

%
% Include here experiment dependent operation procedures
%
%\newpage
\section{Special Procedures for \EXPTS}
Each shift requires a shift leader and a polarized 3He target operator. A third person on shift is extremely
useful, but not required. The shift leader has the standard duties of shift leader to ensure proper data taking,
log all activity, and to fill out the Beam Accounting form. The target operator should focus on the operation
of the polarized 3He target. Target operator training is arranged by Gordon Cates. The shift leader or third
shift worker will run the DAQ and online analysis codes, and fill out the shift checklist. This run period
uses the standard Hall A equipment, BigBite Spectrometer, SuperBite Spectrometer and a polarized 3 He
target. The use and safety procedures of the standard equipment are documented in the Hall A Standard
Equipment Manual available from the Hall A web page. The use and safety procedures of the polarized
3He target are documented in the Target OSP (Target Operation Safety Procedures). The Target OSP and
LOSP are available from the Hall A web page. The use and safety procedures for the equipment of the
BigBite Spectrometer and SuperBite Spectrometer are in various OSPs that are available from the Hall A
web page. One special requirement of the polarized 3He target is the “laser lockdown” period. During this
period, only personnel with polarized 3He target training or laser safety training will be allowed to access
the hall. Details of the “laser lockdown” period and its safety and training requirement are documented in
the Target OSP. To summarize, the following OSPs are associated with Experiment \EXPTS.

\begin{itemize}
\item Jefferson Lab \HALL\ Standard Equipment Manual
\item Polarized 3He Target LOSP
\item Polarized 3He Target OSP
\end{itemize}

\newpage
\section{Signature Sheets}

After reading this document, as well as the ESAD, RSAD, and ERG, workers need to sign
the signature sheet located in the ``yellow binder'' of the experiment specific documents.
This binder can be found in the \HALL\ counting house and in the MCC.

